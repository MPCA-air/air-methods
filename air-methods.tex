\documentclass[14pt,]{book}
\usepackage{lmodern}
\usepackage{amssymb,amsmath}
\usepackage{ifxetex,ifluatex}
\usepackage{fixltx2e} % provides \textsubscript
\ifnum 0\ifxetex 1\fi\ifluatex 1\fi=0 % if pdftex
  \usepackage[T1]{fontenc}
  \usepackage[utf8]{inputenc}
\else % if luatex or xelatex
  \ifxetex
    \usepackage{mathspec}
  \else
    \usepackage{fontspec}
  \fi
  \defaultfontfeatures{Ligatures=TeX,Scale=MatchLowercase}
    \setmonofont[Mapping=tex-ansi,Scale=0.7]{Source Code Pro}
\fi
% use upquote if available, for straight quotes in verbatim environments
\IfFileExists{upquote.sty}{\usepackage{upquote}}{}
% use microtype if available
\IfFileExists{microtype.sty}{%
\usepackage{microtype}
\UseMicrotypeSet[protrusion]{basicmath} % disable protrusion for tt fonts
}{}
\usepackage[margin=1in]{geometry}
\usepackage{hyperref}
\hypersetup{unicode=true,
            pdftitle={Methods for air data analysis},
            pdfborder={0 0 0},
            breaklinks=true}
\urlstyle{same}  % don't use monospace font for urls
\usepackage{natbib}
\bibliographystyle{apalike}
\usepackage{longtable,booktabs}
\usepackage{graphicx,grffile}
\makeatletter
\def\maxwidth{\ifdim\Gin@nat@width>\linewidth\linewidth\else\Gin@nat@width\fi}
\def\maxheight{\ifdim\Gin@nat@height>\textheight\textheight\else\Gin@nat@height\fi}
\makeatother
% Scale images if necessary, so that they will not overflow the page
% margins by default, and it is still possible to overwrite the defaults
% using explicit options in \includegraphics[width, height, ...]{}
\setkeys{Gin}{width=\maxwidth,height=\maxheight,keepaspectratio}
\IfFileExists{parskip.sty}{%
\usepackage{parskip}
}{% else
\setlength{\parindent}{0pt}
\setlength{\parskip}{6pt plus 2pt minus 1pt}
}
\setlength{\emergencystretch}{3em}  % prevent overfull lines
\providecommand{\tightlist}{%
  \setlength{\itemsep}{0pt}\setlength{\parskip}{0pt}}
\setcounter{secnumdepth}{5}
% Redefines (sub)paragraphs to behave more like sections
\ifx\paragraph\undefined\else
\let\oldparagraph\paragraph
\renewcommand{\paragraph}[1]{\oldparagraph{#1}\mbox{}}
\fi
\ifx\subparagraph\undefined\else
\let\oldsubparagraph\subparagraph
\renewcommand{\subparagraph}[1]{\oldsubparagraph{#1}\mbox{}}
\fi

%%% Use protect on footnotes to avoid problems with footnotes in titles
\let\rmarkdownfootnote\footnote%
\def\footnote{\protect\rmarkdownfootnote}

%%% Change title format to be more compact
\usepackage{titling}

% Create subtitle command for use in maketitle
\newcommand{\subtitle}[1]{
  \posttitle{
    \begin{center}\large#1\end{center}
    }
}

\setlength{\droptitle}{-2em}
  \title{Methods for air data analysis}
  \pretitle{\vspace{\droptitle}\centering\huge}
  \posttitle{\par}
  \author{}
  \preauthor{}\postauthor{}
  \predate{\centering\large\emph}
  \postdate{\par}
  \date{Updated Jun 07, 2017}

\usepackage{booktabs}

\usepackage{amsthm}
\newtheorem{theorem}{Theorem}[chapter]
\newtheorem{lemma}{Lemma}[chapter]
\theoremstyle{definition}
\newtheorem{definition}{Definition}[chapter]
\newtheorem{corollary}{Corollary}[chapter]
\newtheorem{proposition}{Proposition}[chapter]
\theoremstyle{definition}
\newtheorem{example}{Example}[chapter]
\theoremstyle{remark}
\newtheorem*{remark}{Remark}
\begin{document}
\maketitle

{
\setcounter{tocdepth}{1}
\tableofcontents
}
\chapter*{Introduction}\label{introduction}
\addcontentsline{toc}{chapter}{Introduction}

This guide describes the methods used at the MPCA to analyze air
monitoring and modeling data. The charts and code found in this guide
were produced using the freely available R statistical software. To
follow along with the examples, you can download a copy of R to your
computer from the \href{https://cran.r-project.org/}{r-project} or open
\href{http://www.r-fiddle.org/\#/}{R-Fiddle} in your browser.

\section*{Edit this document}\label{edit-this-document}
\addcontentsline{toc}{section}{Edit this document}

This guide lives on GitHub. To suggest edits go to
\href{https://github.com/MPCA-air/air-methods/issues}{github.com/MPCA-air/air-methods/issues}
and click on \textbf{New issue}.

\begin{enumerate}
\def\labelenumi{\arabic{enumi}.}
\setcounter{enumi}{-1}
\item
\begin{verbatim}
Data cleaning  
\end{verbatim}

  \begin{enumerate}
  \def\labelenumii{\alph{enumii}.}
  \tightlist
  \item
    Duplicates\\
  \item
    POCs
  \end{enumerate}
\end{enumerate}

\begin{enumerate}
\def\labelenumi{\alph{enumi}.}
\item
\begin{verbatim}
Detection limits  
  - Below detection 
  - Multiple detection limits  
\end{verbatim}
\item
\begin{verbatim}
Confidence limits  
\end{verbatim}
\item
\begin{verbatim}
Site comparisons  
\end{verbatim}
\item
\begin{verbatim}
Pollution rose  
i.    openair  
\end{verbatim}
\item
\begin{verbatim}
Calendar plots  
\end{verbatim}
\item
\begin{verbatim}
 Boxplots  
i.    Log  
ii.   Outliers  
\end{verbatim}
\item
\begin{verbatim}
Time series trends  
i.    Seasonality  
ii.    Year to year  
\end{verbatim}
\item
\begin{verbatim}
Change points (before & after intervention)  
\end{verbatim}
\item
\begin{verbatim}
  Maps  
i.    Zonal averaging  
ii.    Kriging  
\end{verbatim}
\end{enumerate}

\chapter{Data cleaning}\label{data-cleaning}

\protect\hyperlink{start}{Start}

\hypertarget{start}{\section{Duplicate obersvations}\label{start}}

You can label chapter and section titles using \texttt{\{\#label\}}
after them, e.g., we can reference \ref{start}. If you do not manually
label them, there will be automatic labels anyway, e.g., \ref{methods}.

Figures and tables with captions will be placed in \texttt{figure} and
\texttt{table} environments, respectively.

You can write citations, too. For example, we are using the
\textbf{bookdown} package \citep{R-bookdown} in this sample book, which
was built on top of R Markdown and \textbf{knitr} \citep{xie2015}. ddd
aaa

\section{Multiple monitors (POCs)}\label{multiple-monitors-pocs}

\chapter{Detection limits}\label{detection-limits}

Here is a review of existing methods.

\section{Below detection}\label{below-detection}

\section{Multiple detection limits}\label{multiple-detection-limits}

\chapter{Confidence limits}\label{confidence-limits}

We describe our methods in this chapter.

\chapter{Site comparisons}\label{site-comparisons}

Some \emph{significant} applications are demonstrated in this chapter.

\section{Example one}\label{example-one}

\chapter{Wind \& pollution roses}\label{wind-pollution-roses}

We have finished a nice book.

\section{\texorpdfstring{\emph{openair}
package}{openair package}}\label{openair-package}

\bibliography{packages,book}


\end{document}
